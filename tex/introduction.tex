% !TeX root = cshep2023.tex

\section*{Introduction}

\begin{frame}{Course timetable and communication}
  \begin{itemize}
  \item Usually Monday 9--11

    \begin{itemize}
    \item Changes are possible, especially with the Thursday 9--11 slot (Module
      5 by Andrea Chierici)
    \item They will be communicated in due time
    \end{itemize}
  \item For any question, comment, clarification, \ldots just contact me via
    e-mail or Teams chat
  \end{itemize}
\end{frame}

\begin{frame}{Supporting material}

  \begin{itemize}

  \item The source of this and other presentations and code examples are kept at
    \url{https://github.com/giacomini/cshep2024}

  \item The presentations will be made available in pdf format on Virtuale and
    possibly on GitHub itself

  \item Suggested resources for \Cpp{}
    \begin{itemize}

    \item \textit{Introduction to \Cpp{}}, available on Virtuale: an
      introductory course on programming I give at the degree in Physics

    \item \textit{Learn \Cpp{}}, \url{https://learncpp.com/}: online resource
      with tutorials, examples, exercises
  
    \item \Cpp{} reference, \url{https://cppreference.com/}: excellent
      reference, though somewhat formal, for the language and the library, with
      examples
  
    \item
      \href{https://isocpp.github.io/CppCoreGuidelines/CppCoreGuidelines}{\textit{\Cpp{}
          Core Guidelines}}: online collection of guidelines for the correct use
      of \Cpp{}
  
    \item B.~Stroustrup,
      \href{https://stroustrup.com/programming.html}{\textit{Programming:
          Principles and Practice Using C++}}, $3^{rd}$ edition, Addison-Wesley

    \end{itemize}

  \item Other resources will be provided during the course
  \end{itemize}

\end{frame}

\begin{frame}{Exam}

  The exam consists of two parts:

  \begin{enumerate}
  \item Development of a parallel program in C++ to be prepared and executed in
    a Linux virtual machine instantiated on the Cloud
  \item An oral part discussing the project and the concepts and technologies
    that are the subject of the course
  \end{enumerate}

  Details will follow.

\end{frame}

\begin{frame}{Platforms and tools}
  \begin{itemize}[<+->]
  \item The reference platform is a recent distribution of Linux with the
    \code{gcc} compiler suite
  \item We can give some help to install and configure:
    \begin{itemize}[<.->]
    \item Window Subsystem for Linux on Windows
    \item XCode (-tools) with \code{gcc} on macOS
    \end{itemize}
  \item As text editor we recommend
    \href{https://code.visualstudio.com/}{\textbf{Visual Studio Code}}
  \item But any text editor is fine: nano, vi, emacs, \ldots
  \item You can experiment with online compilers, in particular
    \href{https://godbolt.org/}{\textbf{Compiler Explorer}}
  \end{itemize}
\end{frame}

\begin{frame}{Course description}

  \begin{itemize}
  \item Elements of computer architecture and operating systems
  \item Application of \Cpp{} programming techniques to scientific software
    development, including data abstraction, polymorphism, generic programming,
    concurrency and parallelism
  \item Use of modern \Cpp{} to safely and efficiently exploit the memory
    hierarchy and the heterogeneous nature of current computer architectures
  \item Introduction to elements of software engineering and use of effective
    development tools
  \end{itemize}

\end{frame}
